\documentclass{article}

\usepackage{amsmath}
\usepackage{amsfonts}
\usepackage{amsthm}
\usepackage{enumitem}
\usepackage{dsfont}
\usepackage[margin=1in]{geometry}
\usepackage{amssymb}
\usepackage{pifont}
\usepackage{mathtools}

\theoremstyle{plain}
\newtheorem{thm}{Theorem}
\newtheorem{lem}{Lemma}
\newtheorem{prop}{Proposition}
\newtheorem{conj}{Conjecture}
\newtheorem{cor}{Corollary}

\theoremstyle{definition}
\newtheorem{claim}{Claim}
\newtheorem{fact}{Fact}
\newtheorem{defn}{Definition}

\newcommand{\scn}[1]{\section{#1}}
\newcommand{\sscn}[1]{\subsection{#1}}
\newcommand{\beq}{\begin{equation}}
\newcommand{\eeq}{\end{equation}}
\newcommand{\beqn}{\begin{equation*}}
\newcommand{\eeqn}{\end{equation*}}
\newcommand{\bmat}{\begin{bmatrix}}
\newcommand{\emat}{\end{bmatrix}}


\newcommand{\N}{\ensuremath{\mathbb{N}}}
\newcommand{\Z}{\ensuremath{\mathbb{Z}}}
\newcommand{\Q}{\ensuremath{\mathbb{Q}}}
\newcommand{\R}{\ensuremath{\mathbb{R}}}
\newcommand{\C}{\ensuremath{\mathbb{C}}}
\newcommand{\F}{\ensuremath{\mathbb{F}}}

\newcommand{\E}{\ensuremath{\mathbb{E}}}
\newcommand{\1}{\ensuremath{\mathds{1}}}

\DeclareMathOperator{\Gal}{Gal}
\DeclareMathOperator{\Jac}{Jac}
\DeclareMathOperator{\Var}{Var}
\DeclareMathOperator{\Cov}{Cov}
\DeclareMathOperator{\Div}{Div}

\title{Sprout 2014:\\
Abstract Algebra, Day 1}
\author{Lou Gaudet and Teddy Weisman}
\date{July 12, 2014}

\begin{document}

\maketitle

\section*{Introduction}

\begin{itemize}

\item Start with a 5-10 minute discussion of what the class already
  knows. Ask students what they think the following words mean:

  \begin{itemize}
  \item Algebra
  \item Symmetry
  \item Geometry
  \item Transformation
  \item Inverse
  \item Structure
  \item Geometry
  \item Equivalence
  \end{itemize}

\item Discuss the style of the math we're doing in this course: we'll
  be learning mostly by asking questions and exploring concepts, but
  as much as possible, we'll try and be explicit and precise about
  seemingly familiar topics.

\item Highlight the contrast between high school algebra they may
  already be familiar with, and \emph{abstract algebra}.

\end{itemize}

\section*{What is a group?}

\begin{itemize}
\item Informally define the concept of a \emph{set}, and encourage
  students to come up with some silly-sounding examples (we can put
  whatever we want between curly braces!)

\item Say what it means for a set to lack \emph{structure} by
  contrasting a set with $\Z$, which has \emph{addition}, a
  \emph{binary operation}. (Define the word \emph{integer} here). 

\item Ask students to list properties of addition. Recognize
  (underline, star, etc.) properties which correspond to \emph{group
    axioms}. Put other properties off to the side.

\item Give the formal definition of a group

\end{itemize}

\section*{Examples and non-examples of groups}
\begin{itemize}
\item $\Z$. (It better satisfy the properties that we picked out from
  it!)

\item The number line ($\R$). Mention $\Q$ and $\C$ for those who have
  heard of them. 

\item All of these groups are infinite, so try and build a
  finite group:

  \begin{itemize}

  \item Nonexample: $A = \{x \in \Z : 0 \le x < 10\}$ (\emph{write:
      ``positive integers less than 10''}).

  \item Introduce ``addition modulo 10'' as a new example of a binary
    operation on the integers. $A$ is now a group under this
    operation!

  \item Show that we can also do this for $\Z/3\Z$ (multiplication
    table). Mention that we can do this for any $n$.
    
  \end{itemize}

\item Groups don't need to be ``numbers'': rotations in $\R^2$

\item Need for an identity element: $\N$

\end{itemize}

\subsection*{Activity: which of the following are groups?}

\begin{itemize}
\item $2\Z$ (yes)
\item $\N \cup \{0\}$ (no)
\item $(\R \setminus \{0\},  \times)$ (yes)
\item Free group on two letters (yes)
\item $(\R, -)$ (no)
\item Rotations of a triangle (yes).
\item Coordinate pairs $(x,y)$ with either $x=0$ or $y=0$ (no, draw a
  picture)
\end{itemize}

\subsection*{Commutativity}
\begin{itemize}
\item Define commutativity (if it hasn't been already). Challenge
  students to identify which group from the previous activity is
  \emph{not} commutative (free group).

\item $D_3$ (symmetries of a triangle). Multiplication table. Mention
  generality of $D_n$. 

\item Fun example: Rubik's cube
\end{itemize}

\section*{Isomorphism}

\begin{itemize}

\item Use the multiplication table for $D_3$ to get the multiplcation
  table for ``rotations of a triangle.'' Compare to the multiplication
  table for $\Z/3\Z$ ($r_i \mapsto i$). 

\item Informally define the concept of \emph{isomorphic groups}. Give
  a more formal definition if it seems appropriate.

\item Discuss ``size''(order) of a group

\item Point out how isomorphisms give us a way to count the number of
  groups there are of a particular order ``up to isomorphism.''
  Counting groups isn't like counting sets, it's like counting
  multiplication tables.

\item If there's time, give an explicit example of two non-isomorphic
  groups of the same order (probably $V_4$ and $\Z/4\Z$)
 
\end{itemize}

\section*{Things to think about for next week}

\begin{itemize}

\item Can you come up with any more examples of groups of order $3$?
  Are they isomorphic to the first two groups we came up with?

\item How could we count the number of groups of a particular order
  (up to isomorphism)?

\item What are some ways to tell if two groups are isomorphic?

\end{itemize}

\end{document}