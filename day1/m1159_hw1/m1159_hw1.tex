\documentclass[10pt, letterpaper]{article}
\usepackage{amsmath}
\usepackage{amsfonts}
\usepackage{amsthm}
\usepackage{enumitem}
\usepackage{dsfont}
\usepackage[margin=1in]{geometry}
\usepackage{amssymb}
\usepackage{pifont}
\usepackage{mathtools}

\newtheorem{thm}{Theorem}
\newtheorem{lem}{Lemma}
\newtheorem{prop}{Proposition}
\newtheorem{conj}{Conjecture}
\newtheorem{cor}{Corollary}

\theoremstyle{definition}
\newtheorem{claim}{Claim}
\newtheorem{fact}{Fact}
\newtheorem{defn}{Definition}

\newcommand{\N}{\ensuremath{\mathbb{N}}}
\newcommand{\Z}{\ensuremath{\mathbb{Z}}}
\newcommand{\Q}{\ensuremath{\mathbb{Q}}}
\newcommand{\R}{\ensuremath{\mathbb{R}}}
\newcommand{\C}{\ensuremath{\mathbb{C}}}
\newcommand{\F}{\ensuremath{\mathbb{F}}}

\newcommand{\E}{\ensuremath{\mathbb{E}}}
\newcommand{\1}{\ensuremath{\mathds{1}}}

\DeclareMathOperator{\Gal}{Gal}
\DeclareMathOperator{\Jac}{Jac}
\DeclareMathOperator{\Var}{Var}
\DeclareMathOperator{\Cov}{Cov}
\DeclareMathOperator{\Div}{Div}

\pagenumbering{gobble}

\begin{document}

\title{Math 1159 -- Abstract Algebra: Homework 1}
\author{Lou Gaudet \& Teddy Weisman \\ 
\url{louis.gaudet@yale.edu} \& \url{theodore.weisman@yale.edu}}
\date{July 12, 2014}
\maketitle

All homeworks for this class are optional (but recommended). Feel free
to email us if you have questions!

\subsection*{Group Axioms}
If $G$ is a group with operation $\star$, then:
\begin{enumerate}
\item \emph{(Closure)}: For any elements $a$ and $b$ in $G$, $a \star
  b$ is also in $G$.

\item \emph{(Associativity)}: For any $a$, $b$, and $c$ in $G$, then
  $a \star (b \star c) = (a \star b) \star c$.

\item \emph{(Identity)}: There is an element $e$ in $G$ (the
  \emph{identity element}) such that for any $a$ in $G$, $e \star a =
  a$ and $a \star e = a$. 

\item \emph{(Invertibility)}: For any element $a$ in $G$, there is an
  element $a^{-1}$ (the \emph{inverse} of $a$) such that $a \star
  a^{-1} = e$ and $a^{-1} \star a = e$.

\end{enumerate}

If, for every $a$ and $b$ in $G$, $a \star b = b \star a$, then we say
the group is \emph{commutative}, or \emph{abelian}. \textbf{However:
  this property is not required for $G$ to be a group!}

\subsection*{Exercises}

\begin{enumerate}

\item Which of the following are groups? For each one that isn't,
  identify which of the four properties it lacks.

\begin{enumerate}

\item Rotations of a circle, with the composition operation. (Include
  the identity rotation; that is, the ``rotation'' consisting of
  leaving the circle unchanged).

\item Reflections of a triangle, with the composition
  operation. (Include the identity reflection). 

\item The set of all odd integers, with addition as the operation.

\item $\{0\}$ (the set containing only $0$), with addition.

\item $\{1\}$, with addition

\item $\{1\}$, with multiplication

\item The set of rational numbers with the operation $\star$, where
  $a \star b = \frac{a + b}{5}$
\end{enumerate}

\item The set of symmetries of a square (called $D_4$) is a group
  under composition.
  \begin{enumerate}
    \item Is this group abelian?

    \item Write out the group table for this group. (This is a little
      tedious, but it will help you get a feel for how this group
      works).
  \end{enumerate}

\item Are the groups defined by the following two tables isomorphic?
  Why or why not?

\begin{figure}[h]
\centering
\begin{tabular} { c|cccc }
 $\star$ & 1 & a & b & c \\ \hline
 1 & 1 & a & b & c \\
 a & a & b & c & 1 \\
 b & b & c & 1 & a \\
 c & c & 1 & a & b
\end{tabular}
\qquad
\begin{tabular}{c|cccc}
 $\diamond$ & 1 & a & b & c \\ \hline
1 & 1 & a & b & c \\
a & a & 1 & c & b \\
b & b & c & 1 & a \\
c & c & b & a & 1
\end{tabular}
\end{figure}

\item Can you find any more examples of groups of order $3$? For each
  example you find, write down a group table and check if it is
  isomorphic to one of the groups of order $3$ we found in class.

  \textbf{Challenge}: Can you write down a list of groups (with their
  group tables) so that \emph{any} group of order $3$ is isomorphic to
  something in your list? (\textbf{Hint:} there might not be as many as
  you think!)

\end{enumerate}

\end{document}